% !TeX root = tcolorbox.tex
% include file of tcolorbox.tex (manual of the LaTeX package tcolorbox)
\begin{center}
\vspace*{5mm}
\begin{tcolorbox}[enhanced,
  center upper,width=9cm,boxrule=0.4pt,
  colback=white,colframe=black!50!yellow,drop fuzzy midday shadow=black!50!yellow]
{\bfseries\LARGE The \texttt{tcolorbox} package\par}\medskip
{\large Manual for version \version\ (\datum)\par}
\end{tcolorbox}\bigskip
{\large Thomas F.~Sturm%
  \footnote{Prof.~Dr.~Dr.~Thomas F.~Sturm, Institut f\"{u}r Mathematik und Informatik,
    Universit\"{a}t der Bundeswehr M\"{u}nchen, D-85577 Neubiberg, Germany;
     email: \href{mailto:thomas.sturm@unibw.de}{thomas.sturm@unibw.de}} }
\end{center}
\bigskip
\begin{absquote}
  \begin{center}\bfseries Abstract\end{center}
  |tcolorbox| provides an environment for colored and framed text boxes with a
  heading line. Optionally, such a box can be split in an upper and a lower
  part. The package |tcolorbox| can be used for the setting of \LaTeX\ examples where
  one part of the box displays the source code and the other part shows the
  output. Another common use case is the setting of theorems. The package supports
  saving and reuse of source code and text parts.
\end{absquote}

\begin{tcolorbox}[breakable,enhanced,title={Contents},fonttitle=\bfseries\Large,
  colback=yellow!10!white,colframe=red!50!black,before=\par\bigskip\noindent,
  watermark color=yellow!75!red!25!white,pad at break=3mm,
  watermark text={\bfseries\Large Contents},
  %enlargepage=2\baselineskip,
  drop fuzzy shadow]
\makeatletter
\@starttoc{toc}
\makeatother
\end{tcolorbox}
