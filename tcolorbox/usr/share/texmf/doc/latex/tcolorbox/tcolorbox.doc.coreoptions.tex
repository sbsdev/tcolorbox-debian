% !TeX root = tcolorbox.tex
% include file of tcolorbox.tex (manual of the LaTeX package tcolorbox)
\clearpage
\section{Option Keys}\label{sec:optkeys}
For the \meta{options} in \refEnv{tcolorbox} respectively \refCom{tcbset}
the following |pgf| keys can be applied. The key tree path |/tcb/| is not to
be used inside these macros. It is easy to add your own style keys using
the syntax for |pgf| keys, see \cite{tantau:2010c,sturm:2012k} or the examples
starting from page \pageref{sec:latextutorial}.


\subsection{Title}
\begin{docTcbKey}{title}{=\meta{text}}{no default, initially empty}
  Creates a heading line with \meta{text} as content.
\begin{dispExample*}{sbs,lefthand ratio=0.6}
\begin{tcolorbox}[title=My heading line]
This is a \textbf{tcolorbox}.
\end{tcolorbox}
\end{dispExample*}
\end{docTcbKey}

\begin{docTcbKey}{notitle}{}{no value, initially set}
  Removes the title line if set before.
\end{docTcbKey}


\begin{docTcbKey}{adjusted title}{=\meta{text}}{style, no default, initially unset}
  Creates a heading line with \meta{text} as content. The minimal height of
  this line is adjusted to fit the text given by \refKey{/tcb/adjust text}.
  This option makes sense
  for single line headings if boxes are set side by side with equal height.
  Note that it is very easy to trick this adjustment.
\begin{dispExample}
\tcbset{colback=White,arc=0mm,width=(\linewidth-4pt)/4,
equal height group=AT,before=,after=\hfill,fonttitle=\bfseries}

The following titles are not adjusted:\\
\foreach \n in {xxx,ggg,AAA,\"Agypten}
{\begin{tcolorbox}[title=\n,colframe=red!75!black]
  Some content.\end{tcolorbox}}
Now, we try again with adjusted titles:\\
\foreach \n in {xxx,ggg,AAA,\"Agypten}
{\begin{tcolorbox}[adjusted title=\n,colframe=blue!75!black]
  Some content.\end{tcolorbox}}
\end{dispExample}
\end{docTcbKey}


\begin{docTcbKey}{adjust text}{=\meta{text}}{no default, initially \texttt{\"Apgjy}}
  This sets the reference text for \refKey{/tcb/adjusted title}. If your texts
  never exceed '\"Apgjy' in depth and height you don't need to care about this option.
\end{docTcbKey}



\clearpage
\subsection{Lower Part}
\begin{docTcbKey}{lowerbox}{=\meta{mode}}{no default, initially \texttt{visible}}
  Controls the treatment of the lower part of the box.
  Feasible values for \meta{mode} are:
  \begin{itemize}
  \item |visible|: usual type setting of the lower part,
  \item |invisible|: empty space instead of the lower part contents,
  \item |ignored|: the lower part is not used (here).
  \end{itemize}
  The last two values are usually applied in connection with |savelowerto|.
\begin{dispExample}
\begin{tcolorbox}[lowerbox=invisible,colback=white]
This is a \textbf{tcolorbox}.
\tcblower
This is the lower part (but invisible).
\end{tcolorbox}

\begin{tcolorbox}[lowerbox=ignored,colback=white]
This is a \textbf{tcolorbox}.
\tcblower
This is the lower part (but ignored).
\end{tcolorbox}
\end{dispExample}
\end{docTcbKey}


\begin{docTcbKey}{savelowerto}{=\meta{file name}}{no default, initially empty}
  Saves the content of the lower part in a file for an optional later usage.
\begin{dispExample}
\begin{tcolorbox}[lowerbox=invisible,savelowerto=\jobname_bspsave.tex,colback=white]
This is a \textbf{tcolorbox}.
\tcblower
This is the lower part which may be quite complex:
$\displaystyle f(x)=\frac{1+x^2}{1-x^2}$.
\end{tcolorbox}

Now, we load the saved text:\\
\input{\jobname_bspsave.tex}
\end{dispExample}
\end{docTcbKey}

\clearpage
\begin{docTcbKey}{lower separated}{\colOpt{=\meta{boolean value}}}{default |true|, initially |true|}
If set to |true|, the lower part is visually separated from the upper part.
It depends from the chosen skin how the visualization of the separation is done.

\begin{dispExample}
% \tcbuselibrary{skins}
\tcbset{colback=red!5!white,colframe=red!75!black,fonttitle=\bfseries,nobeforeafter,
  width=(\linewidth-4mm)/2,fontlower=\itshape}

\begin{tcolorbox}[title=Lower separated]
This is the upper part.
\tcblower
This is the lower part.
\end{tcolorbox}\hfill
\begin{tcolorbox}[title=Lower not separated,lower separated=false]
This is the upper part.
\tcblower
This is the lower part.
\end{tcolorbox}
\par\bigskip\par
\begin{tcolorbox}[sidebyside,title=Lower separated]
This is the upper part.
\tcblower
This is the lower part.
\end{tcolorbox}\hfill
\begin{tcolorbox}[sidebyside,title=Lower not separated,lower separated=false]
This is the upper part.
\tcblower
This is the lower part.
\end{tcolorbox}
\par\bigskip\par
\begin{tcolorbox}[beamer,title=Lower separated]
This is the upper part.
\tcblower
This is the lower part.
\end{tcolorbox}\hfill
\begin{tcolorbox}[beamer,title=Lower not separated,lower separated=false]
This is the upper part.
\tcblower
This is the lower part.
\end{tcolorbox}
\end{dispExample}
\end{docTcbKey}


\clearpage
\begin{docTcbKey}{savedelimiter}{=\meta{name}}{no default, initially \texttt{tcolorbox}}
  Used in connection with new environment definitions which extend
  |tcolorbox| and use or allow the option |savelowerto|.
  To catch the end of the new box environment \meta{name} has to be the name of
  this environment. Additionally, the environment definition has to use
  |\tcolorbox| instead of
  |\begin{tcolorbox}| and |\endtcolorbox| instead of |\end{tcolorbox}|.
\begin{dispExample}
\newenvironment{mybox}[1]{%
  \tcolorbox[savedelimiter=mybox,
             savelowerto=\jobname_bspsave2.tex,lowerbox=ignored,
             colback=red!5!white,colframe=red!75!black,fonttitle=\bfseries,
             title=#1]}%
  {\endtcolorbox}

\begin{mybox}{My Example}
Upper part.
\tcblower
Saved lower part!
\end{mybox}

Now, the saved part is used:
\begin{tcolorbox}[colback=green!5]
\input{\jobname_bspsave2.tex}
\end{tcolorbox}
\end{dispExample}

The |savedelimiter| is used implicitely with \refCom{newtcolorbox} which
allows a more convenient usage:
\begin{dispExample}
\newtcolorbox{mybox}[1]{%
             savelowerto=\jobname_bspsave2.tex,lowerbox=ignored,
             colback=red!5!white,colframe=red!75!black,fonttitle=\bfseries,
             title=#1}%

\begin{mybox}{My Example}
Upper part.
\tcblower
Saved lower part!
\end{mybox}

Now, the saved part is used:
\begin{tcolorbox}[colback=green!5]
\input{\jobname_bspsave2.tex}
\end{tcolorbox}
\end{dispExample}
\end{docTcbKey}



\clearpage
\subsection{Colors and Fonts}
\begin{docTcbKey}{colback}{=\meta{color}}{no default, initially \texttt{black!5!white}}
  Sets the background \meta{color} of the box.
\begin{dispExample*}{sbs,lefthand ratio=0.6}
\begin{tcolorbox}[colback=white]
This is a \textbf{tcolorbox}.
\end{tcolorbox}
\end{dispExample*}
\end{docTcbKey}


\begin{docTcbKey}{colframe}{=\meta{color}}{no default, initially \texttt{black!75!white}}
  Sets the frame \meta{color} of the box.
\begin{dispExample*}{sbs,lefthand ratio=0.6}
\begin{tcolorbox}[colframe=red!50!yellow]
This is a \textbf{tcolorbox}.
\end{tcolorbox}
\end{dispExample*}
\end{docTcbKey}

\begin{docTcbKey}{colupper}{=\meta{color}}{no default, initially \texttt{black}}
  Sets the text \meta{color} of the upper part.
\begin{dispExample*}{sbs,lefthand ratio=0.6}
\begin{tcolorbox}[colupper=yellow,
  colback=blue!50,colframe=blue]
This is a \textbf{tcolorbox}.
\tcblower
This is the lower part.
\end{tcolorbox}
\end{dispExample*}
\end{docTcbKey}


\begin{docTcbKey}{collower}{=\meta{color}}{no default, initially \texttt{black}}
  Sets the text \meta{color} of the lower part.
\begin{dispExample*}{sbs,lefthand ratio=0.6}
\begin{tcolorbox}[collower=yellow,
  colback=blue!50,colframe=blue]
This is a \textbf{tcolorbox}.
\tcblower
This is the lower part.
\end{tcolorbox}
\end{dispExample*}
\end{docTcbKey}


\begin{docTcbKey}{coltext}{=\meta{color}}{style, no default, initially \texttt{black}}
  Sets the text \meta{color} of the box. This is an abbreviation for setting
  |colupper| and |collower| to the same value.
\begin{dispExample*}{sbs,lefthand ratio=0.6}
\begin{tcolorbox}[coltext=yellow,
  colback=blue!50,colframe=blue]
This is a \textbf{tcolorbox}.
\tcblower
This is the lower part.
\end{tcolorbox}
\end{dispExample*}
\end{docTcbKey}


\begin{docTcbKey}{coltitle}{=\meta{color}}{no default, initially \texttt{white}}
  Sets the title text \meta{color} of the box.
\begin{dispExample*}{sbs,lefthand ratio=0.6}
\begin{tcolorbox}[coltitle=blue!50!black,
  colframe=blue!25,title=Test]
This is a \textbf{tcolorbox}.
\end{tcolorbox}
\end{dispExample*}
\end{docTcbKey}

\clearpage
\begin{docTcbKey}{fontupper}{=\meta{text}}{no default, initially empty}
  Sets \meta{text} before the content of the upper part (e.\,g.\ font settings).
\begin{dispExample}
\begin{tcolorbox}[fontupper=Hello!~\sffamily]
This is a \textbf{tcolorbox}.
\end{tcolorbox}
\end{dispExample}
\end{docTcbKey}


\begin{docTcbKey}{fontlower}{=\meta{text}}{no default, initially empty}
  Sets \meta{text} before the content of the lower part (e.\,g.\ font settings).
\begin{dispExample}
\begin{tcolorbox}[fontlower=\sffamily\bfseries]
This is a \textbf{tcolorbox}.
\tcblower
This is the lower part.
\end{tcolorbox}
\end{dispExample}
\end{docTcbKey}


\begin{docTcbKey}{fonttitle}{=\meta{text}}{no default, initially empty}
  Sets \meta{text} before the content of the title text (e.\,g.\ font settings).
\begin{dispExample}
\begin{tcolorbox}[fonttitle=\sffamily\bfseries\large,title=Hello]
This is a \textbf{tcolorbox}.
\end{tcolorbox}
\end{dispExample}
\end{docTcbKey}

\bigskip
\begin{marker}
  More color options are provided by using skins documented in
  Section \ref{sec:skins} from page \pageref{sec:skins}.
\end{marker}

\clearpage
\subsection{Geometry}
\begin{docTcbKey}{width}{=\meta{length}}{no default, initially \cs{linewidth}}
  Sets the total width of the colored box to \meta{length}.
  See also \refKey{/tcb/height}.
\begin{dispExample}
\tcbset{colback=red!5!white,colframe=red!75!black}

\begin{tcolorbox}[width=\linewidth/2]
This is a \textbf{tcolorbox}.
\end{tcolorbox}
\end{dispExample}
\end{docTcbKey}


\begin{docTcbKey}{toprule}{=\meta{length}}{no default, initially \texttt{0.5mm}}
  Sets the line width of the top rule to \meta{length}.
\begin{dispExample}
\tcbset{colback=red!5!white,colframe=red!75!black}

\begin{tcolorbox}[toprule=3mm]
This is a \textbf{tcolorbox}.
\end{tcolorbox}
\end{dispExample}
\end{docTcbKey}


\begin{docTcbKey}{bottomrule}{=\meta{length}}{no default, initially \texttt{0.5mm}}
  Sets the line width of the bottom rule to \meta{length}.
\begin{dispExample}
\tcbset{colback=red!5!white,colframe=red!75!black}

\begin{tcolorbox}[bottomrule=3mm]
This is a \textbf{tcolorbox}.
\end{tcolorbox}
\end{dispExample}
\end{docTcbKey}


\begin{docTcbKey}{leftrule}{=\meta{length}}{no default, initially \texttt{0.5mm}}
  Sets the line width of the left rule to \meta{length}.
\begin{dispExample}
\tcbset{colback=red!5!white,colframe=red!75!black}

\begin{tcolorbox}[leftrule=3mm]
This is a \textbf{tcolorbox}.
\end{tcolorbox}
\end{dispExample}
\end{docTcbKey}

\clearpage
\begin{docTcbKey}{rightrule}{=\meta{length}}{no default, initially \texttt{0.5mm}}
  Sets the line width of the right rule to \meta{length}.
\begin{dispExample}
\tcbset{colback=red!5!white,colframe=red!75!black}

\begin{tcolorbox}[rightrule=3mm]
This is a \textbf{tcolorbox}.
\end{tcolorbox}
\end{dispExample}
\end{docTcbKey}

\begin{docTcbKey}{titlerule}{=\meta{length}}{no default, initially \texttt{0.5mm}}
  Sets the line width of the rule below the title to \meta{length}.
\begin{dispExample}
\tcbset{enhanced,colback=red!5!white,colframe=red!75!black,
  colbacktitle=red!90!black}

\begin{tcolorbox}[titlerule=3mm,title=This is the title]
This is a \textbf{tcolorbox}.
\end{tcolorbox}
\end{dispExample}
\end{docTcbKey}


\begin{docTcbKey}{boxrule}{=\meta{length}}{style, no default, initially \texttt{0.5mm}}
  Sets all rules of the frame to \meta{length}, i.\,e.\ 
  \refKey{/tcb/toprule}, \refKey{/tcb/bottomrule}, \refKey{/tcb/leftrule},
  \refKey{/tcb/rightrule}, and \refKey{/tcb/titlerule}.
\begin{dispExample}
\tcbset{colback=red!5!white,colframe=red!75!black}

\begin{tcolorbox}[boxrule=3mm]
This is a \textbf{tcolorbox}.
\end{tcolorbox}
\end{dispExample}
\end{docTcbKey}

\bigskip
\begin{marker}
  More options for drawing a \refKey{/tcb/borderline} are provided by using skins documented in
  Section \ref{sec:skins} from page \pageref{sec:skins}.
\end{marker}


\clearpage
\begin{docTcbKey}{arc}{=\meta{length}}{no default, initially \texttt{1mm}}
  Sets the inner radius of the four frame arcs to \meta{length}.
\begin{dispExample}
\tcbset{colback=red!5!white,colframe=red!75!black}

\begin{tcolorbox}[arc=0mm]
This is a \textbf{tcolorbox}.
\end{tcolorbox}
\begin{tcolorbox}[arc=4mm]
This is a \textbf{tcolorbox}.
\end{tcolorbox}
\end{dispExample}
\end{docTcbKey}


\begin{docTcbKey}{outer arc}{=\meta{length}}{no default, initially unset}
  Sets the outer radius of the four frame arcs to \meta{length}.
\begin{dispExample}
\tcbset{colback=red!5!white,colframe=red!75!black}

\begin{tcolorbox}[arc=4mm,outer arc=1mm]
This is a \textbf{tcolorbox}.
\end{tcolorbox}
\end{dispExample}
\end{docTcbKey}


\begin{docTcbKey}{auto outer arc}{}{no value, initially set}
  Sets the outer radius of the four frame arcs automatically in
  dependency of the inner radius given by \refKey{/tcb/arc}.
\end{docTcbKey}


\begin{docTcbKey}{boxsep}{=\meta{length}}{no default, initially \texttt{1mm}}
  Sets a common padding of \meta{length} between the text content and the
  frame of the box. This value is added to the key values of
  |left|, |right|, |top|, |bottom|, and |middle| at the appropriate places.
\begin{dispExample}
\tcbset{colback=red!5!white,colframe=red!75!black,width=(\linewidth-4mm)/2,
  before=,after=\hfill}

\begin{tcolorbox}[boxsep=5mm]
This is a \textbf{tcolorbox}.
\end{tcolorbox}
\begin{tcolorbox}[boxsep=5mm,draft]
This is a \textbf{tcolorbox}.
\end{tcolorbox}
\end{dispExample}
\end{docTcbKey}


\clearpage
\begin{docTcbKey}{left}{=\meta{length}}{style, no default, initially \texttt{4mm}}
  Sets the left space between all text parts and frame (additional to |boxsep|).
  This is an abbreviation for setting
  |lefttitle|, |leftupper|, and |leftlower| to the same value.
\begin{dispExample}
\tcbset{colback=red!5!white,colframe=red!75!black}

\begin{tcolorbox}[left=0mm]
This is a \textbf{tcolorbox}.
\end{tcolorbox}
\end{dispExample}
\end{docTcbKey}

\begin{docTcbKey}{lefttitle}{=\meta{length}}{no default, initially \texttt{4mm}}
  Sets the left space between title text and frame (additional to |boxsep|).
\begin{dispExample}
\tcbset{colback=red!5!white,colframe=red!75!black}

\begin{tcolorbox}[lefttitle=3cm,title=My Title]
This is a \textbf{tcolorbox}.
\end{tcolorbox}
\end{dispExample}
\end{docTcbKey}

\begin{docTcbKey}{leftupper}{=\meta{length}}{no default, initially \texttt{4mm}}
  Sets the left space between upper text and frame (additional to |boxsep|).
\begin{dispExample}
\tcbset{colback=red!5!white,colframe=red!75!black}

\begin{tcolorbox}[leftupper=3cm,title=My Title]
This is a \textbf{tcolorbox}.
\end{tcolorbox}
\end{dispExample}
\end{docTcbKey}

\begin{docTcbKey}{leftlower}{=\meta{length}}{no default, initially \texttt{4mm}}
  Sets the left space between lower text and frame (additional to |boxsep|).
\begin{dispExample}
\tcbset{colback=red!5!white,colframe=red!75!black}

\begin{tcolorbox}[leftlower=3cm]
This is a \textbf{tcolorbox}.
\tcblower
This is the lower part.
\end{tcolorbox}
\end{dispExample}
\end{docTcbKey}


\clearpage
\begin{docTcbKey}{right}{=\meta{length}}{style, no default, initially \texttt{4mm}}
  Sets the right space between all text parts and frame (additional to |boxsep|).
  This is an abbreviation for setting
  |righttitle|, |rightupper|, and |rightlower| to the same value.
\begin{dispExample}
\tcbset{colback=red!5!white,colframe=red!75!black}

\begin{tcolorbox}[width=5cm,right=2cm]
This is a \textbf{tcolorbox}.
\end{tcolorbox}
\end{dispExample}
\end{docTcbKey}



\begin{docTcbKey}{righttitle}{=\meta{length}}{no default, initially \texttt{4mm}}
  Sets the right space between title text and frame (additional to |boxsep|).
\begin{dispExample}
\tcbset{colback=red!5!white,colframe=red!75!black}

\begin{tcolorbox}[width=5cm,righttitle=2cm,title=My very long title text]
This is a \textbf{tcolorbox} with standard upper box dimensions.
\end{tcolorbox}
\end{dispExample}
\end{docTcbKey}


\begin{docTcbKey}{rightupper}{=\meta{length}}{no default, initially \texttt{4mm}}
  Sets the right space between upper text and frame (additional to |boxsep|).
\begin{dispExample}
\tcbset{colback=red!5!white,colframe=red!75!black}

\begin{tcolorbox}[width=5cm,rightupper=2cm,title=My very long title text]
This is a \textbf{tcolorbox} with compressed upper box dimensions.
\end{tcolorbox}
\end{dispExample}
\end{docTcbKey}


\clearpage
\begin{docTcbKey}{rightlower}{=\meta{length}}{no default, initially \texttt{4mm}}
  Sets the right space between lower text and frame (additional to |boxsep|).
\begin{dispExample}
\tcbset{colback=red!5!white,colframe=red!75!black}

\begin{tcolorbox}[width=5cm,rightlower=2cm]
This is a \textbf{tcolorbox} with standard upper box dimensions.
\tcblower
This is the lower part with large space at right.
\end{tcolorbox}
\end{dispExample}
\end{docTcbKey}



\begin{docTcbKey}{top}{=\meta{length}}{no default, initially \texttt{2mm}}
  Sets the top space between text and frame (additional to |boxsep|).
\begin{dispExample}
\tcbset{colback=red!5!white,colframe=red!75!black}

\begin{tcolorbox}[top=0mm]
This is a \textbf{tcolorbox}.
\tcblower
This is the lower part.
\end{tcolorbox}
\end{dispExample}
\end{docTcbKey}


\begin{docTcbKey}{toptitle}{=\meta{length}}{no default, initially \texttt{0mm}}
  Sets the top space between title and frame (additional to |boxsep|).
\begin{dispExample}
\tcbset{colback=red!5!white,colframe=red!75!black}

\begin{tcolorbox}[toptitle=3mm,title=My title]
This is a \textbf{tcolorbox}.
\end{tcolorbox}
\end{dispExample}
\end{docTcbKey}

\clearpage
\begin{docTcbKey}{bottom}{=\meta{length}}{no default, initially \texttt{2mm}}
  Sets the bottom space between text and frame (additional to |boxsep|).
\begin{dispExample}
\tcbset{colback=red!5!white,colframe=red!75!black}

\begin{tcolorbox}[bottom=0mm]
This is a \textbf{tcolorbox}.
\tcblower
This is the lower part.
\end{tcolorbox}
\end{dispExample}
\end{docTcbKey}

\begin{docTcbKey}{bottomtitle}{=\meta{length}}{no default, initially \texttt{0mm}}
  Sets the bottom space between title and frame (additional to |boxsep|).
\begin{dispExample}
\tcbset{colback=red!5!white,colframe=red!75!black}

\begin{tcolorbox}[bottomtitle=3mm,title=My title]
This is a \textbf{tcolorbox}.
\end{tcolorbox}
\end{dispExample}
\end{docTcbKey}


\begin{docTcbKey}{middle}{=\meta{length}}{no default, initially \texttt{2mm}}
  Sets the space between upper and lower text to the separation line
  (additional to |boxsep|).
\begin{dispExample}
\tcbset{colback=red!5!white,colframe=red!75!black}

\begin{tcolorbox}[middle=0mm,boxsep=0mm]
This is a \textbf{tcolorbox}.
\tcblower
This is the lower part.
\end{tcolorbox}
\end{dispExample}
\end{docTcbKey}


\clearpage
\begin{docTcbKey}{oversize}{\colOpt{=\meta{length}}}{style, default |0pt|}
  Sets the text width of the upper part to the current line width plus an
  optional \meta{length}.
  This is achieved by changing the keys \refKey{/tcb/width}
  \refKey{/tcb/enlarge left by}, and
  \refKey{/tcb/enlarge right by} appropriately.
  The resulting box is overlapping into the left and right margin of
  the page.
  Note that this style option has to be given \emph{after} all other
  geometry keys!
\begin{dispListing}
\tcbset{colback=red!5!white,colframe=red!75!black,fonttitle=\bfseries}

\textit{Normal text for comparison:}\\
\lipsum[2]

\begin{tcolorbox}[oversize,title=Oversized box]
\lipsum[2]
\end{tcolorbox}

\begin{tcolorbox}[title=Normal box]
\lipsum[2]
\end{tcolorbox}
\end{dispListing}
\end{docTcbKey}

{\tcbusetemp}


\clearpage
\begin{docTcbKey}{toggle left and right}{=\meta{toggle preset}}{default |evenpage|, initially |none|}
  According to the \meta{toggle preset}, the left and the right settings of
  the |tcolorbox| are switched or not. Feasible values are:
  \begin{itemize}
  \item |none|: no switching.
  \item |forced|: the values of the left and right rules and spaces are switched.
  \item |evenpage|: if the page is an even page, the values of the left and
    right rules and spaces are switched. It is recommended to use this setting in
    conjunction with \refKey{/tcb/check odd page}.
  \end{itemize}
\begin{dispListing}
% \usepackage{changepage} for 'check odd page'
% \usepackage{lipsum}
% \usetikzlibrary{patterns}
% \tcbuselibrary{skins,breakable,hooks}
\begin{tcolorbox}[skin=enhancedmiddle,breakable,
  check odd page,toggle left and right,
  boxrule=0mm,top=0mm,bottom=0mm,left=1mm,right=1mm,
  rightrule=1cm,colupper=blue!25!black,
  watermark graphics=lichtspiel.jpg,watermark opacity=0.5,
  watermark overzoom=1.00,watermark opacity=0.25,
  frame style={pattern=crosshatch dots light steel blue},
  overlay app={%
    \ifoddpage\coordinate (X) at ([xshift=-5mm]frame.east);
         \else\coordinate (X) at ([xshift=5mm]frame.west);\fi
    \fill[shading=ball,ball color=blue!50!white,opacity=0.5] (X) circle (4mm);}]
\lipsum[1-6]
\end{tcolorbox}
\end{dispListing}
\medskip

This example switches a |1cm| thick rule from the left to the right side
depending on the page number. Thereby, the rule is always on the outer side
of the double-sided paper. Additionally, a ball is drawn on the outer side
with help of an overlay.
\bigskip

\tcbusetemp
\end{docTcbKey}

\clearpage
\subsection{Height Control}
In a typical usage scenario, the height of a |tcolorbox| is computed automatically
to fit the content. Nevertheless, the height can be set to a fixed value
or to fit commonly for several boxes, e.\,g. if boxes are set side by side.

\bigskip
\begin{marker}
  The height control keys are only applicable to unbreakable boxes.
  If a box is set to be \refKey{/tcb/breakable}, the height is always
  computed according to the \emph{natural height}.
\end{marker}
\bigskip


\begin{docTcbKey}{natural height}{}{no value, initially set}
  Sets the total height of the colored box to its natural height depending
  on the box content.
\end{docTcbKey}

\begin{docTcbKey}{height}{=\meta{length}}{no default}
  Sets the total height of the colored box to \meta{length} independent
  of the box content. \meta{length} is the minimum height of the box, if
  \refKey{/tcb/height plus} is larger than zero.
\begin{dispExample}
\tcbset{width=(\linewidth-2mm)/3,before=,after=\hfill,
colframe=blue!75!black,colback=white}

\begin{tcolorbox}[height=1cm,valign=center]
  This box has a height of 1cm.
\end{tcolorbox}
\begin{tcolorbox}[height=2cm,valign=center]
  This box has a height of 2cm.
\end{tcolorbox}
\begin{tcolorbox}[height=3cm,split=0.5,valign=center,valign lower=center]
  This box has a height of 3cm.
  \tcblower
  Lower part.
\end{tcolorbox}
\end{dispExample}
\end{docTcbKey}

\enlargethispage*{10mm}
\begin{docTcbKey}{height plus}{=\meta{length}}{no default, initially |0pt|}
  The box may extend a given fixed \refKey{/tcb/height} up to the given \meta{length}.
\begin{dispExample}
\tcbset{colback=red!5!white,colframe=red!75!black,left=1mm,top=1mm,bottom=1mm,
  right=1mm,boxsep=0mm,width=3cm,nobeforeafter}

\begin{tcolorbox}[height=1cm]
This is a tcolorbox.
\end{tcolorbox}
\begin{tcolorbox}[height=1cm,height plus=1cm]
This is a tcolorbox.
\end{tcolorbox}
\begin{tcolorbox}[height=1cm,height plus=1cm]
This is a tcolorbox. This is a tcolorbox. This is a tcolorbox.
\end{tcolorbox}
\end{dispExample}
\end{docTcbKey}


\begin{docTcbKey}{height from}{=\meta{min} \texttt{to} \meta{max}}{style, no default}
  Sets the box height to a dimension between \meta{min} and \meta{max}.
\begin{dispExample}
% \usepackage{lipsum}
\newtcolorbox{mybox}{colback=red!5!white,colframe=red!75!black,left=1mm,top=1mm,
  bottom=1mm,right=1mm,boxsep=0mm,width=4.5cm,nobeforeafter,
  height from=2cm to 8cm}

\begin{mybox}
This is a tcolorbox.
\end{mybox}
\begin{mybox}
This is a tcolorbox. This is a tcolorbox. This is a tcolorbox.
\end{mybox}
\begin{mybox}
\lipsum[2]
\end{mybox}
\end{dispExample}
\end{docTcbKey}



\clearpage
\begin{docTcbKey}{space}{=\meta{fraction}}{no default, initially 0}
  If the height of a |tcolorbox| is not the natural height, the space
  difference between the forced and the natural size is distributed
  between the upper and the lower part of the box. This space could also
  be negative.
  \meta{fraction} with a value between 0 and 1 is the amount of space
  which is added to the upper part, the rest is added to the lower part.
  If there is no lower part, then all of the space is added to
  the upper part always.
\begin{dispExample}
\tcbset{width=(\linewidth-2mm)/3,before=,after=\hfill,
colframe=blue!75!black,colback=white,height=3cm}

\foreach \f in {0.2,0.4,0.7}
{\begin{tcolorbox}[space=\f]
    This is the upper part.
    \tcblower
    This is the lower part.
\end{tcolorbox}}
\end{dispExample}
\end{docTcbKey}

\begin{docTcbKey}{space to upper}{}{style}
  This is an abbreviation for |space=1|, i.\,e. all extra space is added
  to the upper part.
\end{docTcbKey}

\begin{docTcbKey}{space to lower}{}{style, initially set}
  This is an abbreviation for |space=0|, i.\,e. all extra space is added
  to the lower part (if there is any).
\end{docTcbKey}

\begin{docTcbKey}{space to both}{}{style}
  This is an abbreviation for |space=0.5|, i.\,e. the extra space
  equally distributed between the upper and the lower part.
\begin{dispExample}
\tcbset{width=(\linewidth-2mm)/3,before=,after=\hfill,
colframe=blue!75!black,colback=white,height=3cm}

\foreach \myspace in {space to upper,space to both,space to lower}
{\begin{tcolorbox}[\myspace]
    This is the upper part.
    \tcblower
    This is the lower part.
\end{tcolorbox}}
\end{dispExample}
\end{docTcbKey}


\clearpage
\begin{docTcbKey}{split}{=\meta{fraction}}{no default}
  If the height of a |tcolorbox| is not the natural height, the
  \meta{fraction} with a value between 0 and 1 determines the positioning
  of the segmentation between the upper and the lower part. Here, 0 stands
  for top and 1 for bottom. Note that the box is split regardless of
  the actual dimensions of the text parts!
\begin{dispExample}
\tcbset{width=(\linewidth-2mm)/3,before=,after=\hfill,height=3cm,
colback=white,colframe=blue!75!black,valign=center,valign lower=center}

\foreach \f in {0.1,0.5,0.8}
{\begin{tcolorbox}[split=\f]
This is the upper part.
\tcblower
This is the lower part with a lot of text in several lines.
\end{tcolorbox}}
\end{dispExample}
\end{docTcbKey}


\begin{docTcbKey}{valign}{=\meta{alignment}}{no default, initially |top|}
  If the height of a |tcolorbox| is not the natural height, |valign|
  determines the vertical \meta{alignment} of the upper part.
  Feasible values are |top|, |center|, and |bottom|. For a box with natural
  height, these values are meaningless.
\begin{dispExample}
\tcbset{width=(\linewidth-2mm)/3,before=,after=\hfill,
colframe=blue!75!black,colback=white,height=2cm}

\foreach \myalign in {top,center,bottom}
{\begin{tcolorbox}[valign=\myalign]
  This is a \textbf{tcolorbox}.
\end{tcolorbox}}
\end{dispExample}
\end{docTcbKey}


\begin{docTcbKey}{valign lower}{=\meta{alignment}}{no default, initially |top|}
  This key has the same meaning for the lower part as |valign|
  for the upper part, i.\,e. it determines
  the vertical \meta{alignment} of the lower part with feasible values
  |top|, |center|, and |bottom|.
\end{docTcbKey}


\clearpage
\begin{docTcbKey}{equal height group}{=\meta{id}}{no default}
  Boxes which are members of an |equal height group| will all get the
  same height, i.\,e. the maximum of all their natural heights. The
  \meta{id} serves to distinguish between different height groups.
  This \meta{id} should contain only characters which are feasible
  for \TeX\ macro names, typically alphabetic characters but no numerals
  and spaces. Note that you have to compile twice to see changes and
  that height groups are global definitions.


\begin{dispExample}
\tcbset{width=(\linewidth-2mm)/3,before=,after=\hfill,arc=0mm,
colframe=blue!75!black,colback=white,fonttitle=\bfseries}

\begin{tcolorbox}[equal height group=A,adjusted title={One}]
  My smallest box.
\end{tcolorbox}%
\begin{tcolorbox}[equal height group=A,adjusted title={Two}]
  This box is also small.
  \tcblower
  But with a lower part.
\end{tcolorbox}%
\begin{tcolorbox}[equal height group=A,adjusted title={Three}]
  This box contains a lot of text just to fill the space
  with word flowing and flowing and flowing until the box
  is filled with all of it.
\end{tcolorbox}\linebreak
%
\tcbset{width=(\linewidth-1mm)/2,before=,after=\hfill,arc=0mm,
colframe=red!75!black,colback=white}
%
\begin{tcolorbox}[equal height group=B]
  Now, we use another equal height group.
\end{tcolorbox}%
\begin{tcolorbox}[equal height group=B]
  \begin{equation*}
    \int\limits_{0}^{1} x^2 = \frac13.
  \end{equation*}
\end{tcolorbox}
\end{dispExample}
\end{docTcbKey}

\clearpage
\begin{docTcbKey}{minimum for equal height group}{=\meta{id}:\meta{length}}{no default, initially unset}
  Plants a \meta{length} into the equal height group with
  the given \meta{id}. This ensures that the height will not drop below
  \meta{length}. Note that you cannot reduce a computed height value
  by using this key with a small value.
  The difference to applying \refKey{/tcb/height} directly is that the boxes
  are never too small for their content.

\begin{dispExample}
\tcbset{colframe=blue!75!black,colback=white,arc=0mm,
  before=,after=\hfill,fonttitle=\bfseries,left=2mm,right=2mm,
  width=3.5cm,
  equal height group=C,
  minimum for equal height group=C:3.5cm}

\begin{tcolorbox}
  My first box. All boxes will get 3.5cm times 3.5cm
  if the content height is not too large.
\end{tcolorbox}%
\begin{tcolorbox}
  My second box.
  \tcblower
  This is the lower part.
\end{tcolorbox}%
\begin{tcblisting}{}
\textbf{Mixed}
with a listing.
\end{tcblisting}
\begin{tcolorbox}[title={Fourth box}]
  My final box.
\end{tcolorbox}%
\end{dispExample}
\end{docTcbKey}


\clearpage
\subsection{Box Content Additions}\label{subsec:contentadditions}
The following options introduce some arbitrary \meta{code} to the content
of a |tcolorbox|. These additions can be given at the begin or at the end
of the title, the upper part, or the lower part.

\begin{docTcbKey}{before title}{=\meta{code}}{no default, initially unset}
  The given \meta{code} is placed \emph{after} the color and font settings
  and \emph{before} the content of the title.
\begin{dispExample}
\tcbset{before title={\textcolor{yellow}{\large Important:}~},
  colback=red!5!white,colframe=red!75!black,fonttitle=\bfseries}

\begin{tcolorbox}[title=My title]
This is a \textbf{tcolorbox}.
\end{tcolorbox}
\end{dispExample}
\end{docTcbKey}


\begin{docTcbKey}{after title}{=\meta{code}}{no default, initially unset}
  The given \meta{code} is placed \emph{after} the content of the title.
\begin{dispExample}
\tcbset{after title={\hfill\colorbox{Navy}{approved}},
  colback=red!5!white,colframe=red!75!black,fonttitle=\bfseries}

\begin{tcolorbox}[title=My title]
This is a \textbf{tcolorbox}.
\end{tcolorbox}
\end{dispExample}
\end{docTcbKey}


\begin{docTcbKey}{before upper}{=\meta{code}}{no default, initially unset}
  The given \meta{code} is placed \emph{after} the color and font settings
  and \emph{before} the content of the upper part.
\begin{dispExample}
\tcbset{before upper={\textit{The story:}\par},
  colback=red!5!white,colframe=red!75!black,fonttitle=\bfseries}

\begin{tcolorbox}[title=My title]
This is a \textbf{tcolorbox}.
\end{tcolorbox}
\end{dispExample}
\end{docTcbKey}

\clearpage
\begin{docTcbKey}{after upper}{=\meta{code}}{no default, initially unset}
  The given \meta{code} is placed \emph{after} the content of the upper part.
\begin{dispExample}
\tcbset{after upper={\par\hfill\textit{Read more next week}},
  colback=red!5!white,colframe=red!75!black,fonttitle=\bfseries}

\begin{tcolorbox}[title=My title]
This is a \textbf{tcolorbox}.
\end{tcolorbox}
\end{dispExample}
\end{docTcbKey}


\begin{docTcbKey}{before lower}{=\meta{code}}{no default, initially unset}
  The given \meta{code} is placed \emph{after} the color and font settings
  and \emph{before} the content of the lower part.
\begin{dispExample}
\tcbset{before lower=\textit{Behold:~},colback=red!5!white,colframe=red!75!black}

\begin{tcolorbox}
This is a \textbf{tcolorbox}.
\tcblower
This is the lower part.
\end{tcolorbox}
\end{dispExample}
\end{docTcbKey}


\begin{docTcbKey}{after lower}{=\meta{code}}{no default, initially unset}
  The given \meta{code} is placed \emph{after} the content of the lower part.
\begin{dispExample}
\tcbset{after lower=\ \textit{This is the end.},
  colback=red!5!white,colframe=red!75!black}

\begin{tcolorbox}
This is a \textbf{tcolorbox}.
\tcblower
This is the lower part.
\end{tcolorbox}
\end{dispExample}
\end{docTcbKey}


\clearpage
\begin{docTcbKey}{center title}{}{style, no value, initially unset}
  This style sets \refKey{/tcb/before title} to |\centering|.
\begin{dispExample}
\tcbset{center title,colback=red!5!white,colframe=red!75!black,fonttitle=\bfseries}

\begin{tcolorbox}[title=My title]
This is a \textbf{tcolorbox}.
\end{tcolorbox}
\end{dispExample}
\end{docTcbKey}


\begin{docTcbKey}{center upper}{}{style, no value, initially unset}
  This style sets \refKey{/tcb/before upper} to |\centering|.
\begin{dispExample}
\tcbset{center upper,colback=red!5!white,colframe=red!75!black,fonttitle=\bfseries}

\begin{tcolorbox}[title=My title]
This is a \textbf{tcolorbox}.
\end{tcolorbox}
\end{dispExample}
\end{docTcbKey}


\begin{docTcbKey}{center lower}{}{style, no value, initially unset}
  This style sets \refKey{/tcb/before lower} to |\centering|.
\begin{dispExample}
\tcbset{center lower,colback=red!5!white,colframe=red!75!black,fonttitle=\bfseries}

\begin{tcolorbox}[title=My title]
This is a \textbf{tcolorbox}.
\tcblower
This is the lower part.
\end{tcolorbox}
\end{dispExample}
\end{docTcbKey}


\clearpage
\begin{docTcbKey}{flushleft title}{}{style, no value, initially unset}
  This style sets \refKey{/tcb/before title} to |\raggedright|.
\begin{dispExample*}{sbs,lefthand ratio=0.6}
\tcbset{flushleft title,colback=red!5!white,
  colframe=red!75!black,fonttitle=\bfseries}

\begin{tcolorbox}[title=My title which is
  quite long but always flushed left]
This is a \textbf{tcolorbox}.
\end{tcolorbox}
\end{dispExample*}
\end{docTcbKey}


\begin{docTcbKey}{flushleft upper}{}{style, no value, initially unset}
  This style sets \refKey{/tcb/before upper} to |\raggedright|.
\begin{dispExample*}{sbs,lefthand ratio=0.6}
\tcbset{flushleft upper,colback=red!5!white,
  colframe=red!75!black,fonttitle=\bfseries}

\begin{tcolorbox}[title=My title]
This is a \textbf{tcolorbox}. Here, the
content is flushed left.
You see the effect on the right hand side.
\end{tcolorbox}
\end{dispExample*}
\end{docTcbKey}


\begin{docTcbKey}{flushleft lower}{}{style, no value, initially unset}
  This style sets \refKey{/tcb/before lower} to |\raggedright|.
\begin{dispExample*}{sbs,lefthand ratio=0.6}
\tcbset{flushleft lower,colback=red!5!white,
  colframe=red!75!black,fonttitle=\bfseries}

\begin{tcolorbox}[title=My title]
This is a \textbf{tcolorbox}. The upper part
is not flushed left or right. This could be
unpleasant for small boxes.
\tcblower
This is the lower part. Here, the
content is flushed left.
You see the effect on the right hand side.
\end{tcolorbox}
\end{dispExample*}
\end{docTcbKey}


\clearpage
\begin{docTcbKey}{flushright title}{}{style, no value, initially unset}
  This style sets \refKey{/tcb/before title} to |\raggedleft|.
\begin{dispExample*}{sbs,lefthand ratio=0.6}
\tcbset{flushright title,colback=red!5!white,
  colframe=red!75!black,fonttitle=\bfseries}

\begin{tcolorbox}[title=My title]
This is a \textbf{tcolorbox}.
\end{tcolorbox}
\end{dispExample*}
\end{docTcbKey}


\begin{docTcbKey}{flushright upper}{}{style, no value, initially unset}
  This style sets \refKey{/tcb/before upper} to |\raggedleft|.
\begin{dispExample*}{sbs,lefthand ratio=0.6}
\tcbset{flushright upper,colback=red!5!white,
  colframe=red!75!black,fonttitle=\bfseries}

\begin{tcolorbox}[title=My title]
This is a \textbf{tcolorbox}.
\end{tcolorbox}
\end{dispExample*}
\end{docTcbKey}


\begin{docTcbKey}{flushright lower}{}{style, no value, initially unset}
  This style sets \refKey{/tcb/before lower} to |\raggedleft|.
\begin{dispExample*}{sbs,lefthand ratio=0.6}
\tcbset{flushright lower,colback=red!5!white,
  colframe=red!75!black,fonttitle=\bfseries}

\begin{tcolorbox}[title=My title]
This is a \textbf{tcolorbox}.
\tcblower
This is the lower part.
\end{tcolorbox}
\end{dispExample*}
\end{docTcbKey}


\clearpage
\begin{docTcbKey}{tabularx}{=\meta{preamble}}{style}
  This style sets \refKey{/tcb/before upper} and \refKey{/tcb/after upper}
  and several geometry keys to support a |tabularx| with the
  given \meta{preamble}.
  The packages |tabularx| \cite {carlisle:1999a}, |array|, and |colortbl|
  have to be loaded separately.
\begin{dispExample}
% \usepackage{array,tabularx}
% \usepackage{colortbl} - or - \usepackage[table]{xcolor}
\newcolumntype{Y}{>{\raggedleft\arraybackslash}X}% see tabularx
\tcbset{enhanced,fonttitle=\bfseries\large,fontupper=\normalsize\sffamily,
  colback=yellow!10!white,colframe=red!50!black,colbacktitle=Salmon!30!white,
  coltitle=black,center title}

\begin{tcolorbox}[tabularx={X||Y|Y|Y|Y||Y},title=My table]
Group & One     & Two     & Three    & Four     & Sum\\\hline\hline
Red   & 1000.00 & 2000.00 &  3000.00 &  4000.00 & 10000.00\\\hline
Green & 2000.00 & 3000.00 &  4000.00 &  5000.00 & 14000.00\\\hline
Blue  & 3000.00 & 4000.00 &  5000.00 &  6000.00 & 18000.00\\\hline\hline
Sum   & 6000.00 & 9000.00 & 12000.00 & 15000.00 & 42000.00
\end{tcolorbox}
\end{dispExample}
\end{docTcbKey}


\begin{docTcbKey}{tabularx*}{=\marg{code}\marg{preamble}}{style}
  This is a variant of \refKey{/tcb/tabularx} which adds some \meta{code}
  before the table starts.
\begin{dispExample}
% \usepackage{array,tabularx}
% \usepackage{colortbl} - or - \usepackage[table]{xcolor}
\tcbset{enhanced,fonttitle=\bfseries\large,fontupper=\normalsize\sffamily,
  colback=yellow!10!white,colframe=red!50!black,colbacktitle=Salmon!30!white,
  coltitle=black,center title}

\begin{tcolorbox}[tabularx*={\arrayrulewidth0.5mm}{X|X|X},title=My table]
One     & Two     & Three \\\hline\hline
1000.00 & 2000.00 &  3000.00\\\hline
2000.00 & 3000.00 &  4000.00
\end{tcolorbox}
\end{dispExample}
\end{docTcbKey}

\clearpage
\begin{docTcbKey}{tikz upper}{\colOpt{=\meta{options}}}{style}
  This style adds a centered |tikzpicture| environment to the begin and
  to the end of the upper part.
\begin{dispExample}
% \usepackage{tikz}

\begin{tcolorbox}[tikz upper,fonttitle=\bfseries,colback=white,colframe=black,
                  title=TikZ drawing]
  \path[fill=yellow,draw=yellow!75!red] (0,0) circle (1cm);
  \fill[red] (45:5mm) circle (1mm);
  \fill[red] (135:5mm) circle (1mm);
  \draw[line width=1mm,red] (215:5mm) arc (215:325:5mm);
\end{tcolorbox}
\end{dispExample}
\end{docTcbKey}

\begin{docTcbKey}{tikz lower}{\colOpt{=\meta{options}}}{style}
  This style adds a centered |tikzpicture| environment to the begin and
  to the end of the lower part.
\begin{dispExample}
% \usepackage{tikz}
% \usetikzlibrary{skins,listings}
\tcbset{tikz lower,listing side text,fonttitle=\bfseries,
  bicolor,colback=LightBlue!50!white,colbacklower=white,colframe=black,
  righthand width=3cm}

\begin{tcblisting}{title=TikZ drawing}
\path[fill=yellow,draw=yellow!75!red]
    (0,0) circle (1cm);
\fill[red] (45:5mm) circle (1mm);
\fill[red] (135:5mm) circle (1mm);
\draw[line width=1mm,red]
    (215:5mm) arc (215:325:5mm);
\end{tcblisting}
\end{dispExample}
\end{docTcbKey}


\clearpage
\begin{docTcbKey}{code}{=\meta{code}}{no default, initially unset}
  The given \meta{code} is executed immediately. This option is useful
  to place some arbitrary code into an option list.
\begin{dispExample}
\tcbset{colback=red!5!white,colframe=red!75!black,
  code={Useless at this spot but functional.},
  fonttitle=\bfseries}

\begin{tcolorbox}[code={\newcommand{\mycommand}{\textit{working}}},
  title=My \mycommand\ title]
This is a \textbf{tcolorbox}.
\end{tcolorbox}
\end{dispExample}
\end{docTcbKey}

\clearpage
\subsection{Overlays}\label{subsec:overlays}
With an overlay, arbitrary \meta{graphical code} can be added to a
|tcolorbox|. This code is executed \emph{after} the frame and interior are
drawn and \emph{before} the text content is drawn. Therefore, you can
decorate the |tcolorbox| with your own extensions.
Common special cases are \emph{watermarks} which are implemented using overlays.
See Subsection \ref{subsec:watermarks} from page \pageref{subsec:watermarks} if
you want to add \emph{watermarks}.
\begin{marker}
If you use the core package only, the \meta{graphical code} has to be |pgf| code
and there is not much assistance for positioning.
Therefore, the usage of the \refKey{/tcb/enhanced} mode from the library skins
is recommended which allows |tikz| code and gives access to
\refKey{/tcb/geometry nodes} for positioning.
\end{marker}

\begin{docTcbKey}{overlay}{=\meta{graphical code}}{no default, initially unset}
  Adds \meta{graphical code} to the box drawing process. This \meta{graphical code}
  is drawn \emph{after} the frame and interior and \emph{before} the text content.

\begin{dispExample}
% \tcbuselibrary{skins} % preamble
\tcbset{frogbox/.style={enhanced,colback=green!10,colframe=green!65!black,
  enlarge top by=5.5mm,
  overlay={\foreach \x in {2cm,3.5cm} {
    \begin{scope}[shift={([xshift=\x]frame.north west)}]
      \path[draw=green!65!black,fill=green!10,line width=1mm] (0,0) arc (0:180:5mm);
      \path[fill=black] (-0.2,0) arc (0:180:1mm);
    \end{scope}}}]}}

\begin{tcolorbox}[frogbox,title=My title]
This is a \textbf{tcolorbox}.
\end{tcolorbox}
\end{dispExample}

\enlargethispage*{5mm}
\begin{dispExample}
% \usetikzlibrary{patterns} % preamble
% \tcbuselibrary{skins}     % preamble
\tcbset{ribbonbox/.style={enhanced,colback=red!5!white,colframe=red!75!black,
  fonttitle=\bfseries,
  overlay={\path[fill=blue!75!white,draw=blue,double=white!85!blue,
    preaction={opacity=0.6,fill=blue!75!white},
    line width=0.1mm,double distance=0.2mm,
    pattern=fivepointed stars,pattern color=white!75!blue]
    ([xshift=-0.2mm,yshift=-1.02cm]frame.north east)
    -- ++(-1,1) -- ++(-0.5,0) -- ++(1.5,-1.5) -- cycle;}}}

\begin{tcolorbox}[ribbonbox,title=My title]
This is a \textbf{tcolorbox}.
\tcblower
This is the lower part.
\end{tcolorbox}
\end{dispExample}
\end{docTcbKey}

\clearpage
\begin{docTcbKey}{no overlay}{}{style, no default, initially set}
  Removes the overlay if set before.
\end{docTcbKey}

\begin{docTcbKey}{overlay broken}{=\meta{graphical code}}{no default, initially unset}
  If the box is set to be \refKey{/tcb/breakable} and \emph{is} broken actually,
  then the \meta{graphical code} is added to the box drawing process.
  \refKey{/tcb/overlay} overwrites this key.
\end{docTcbKey}

\begin{docTcbKey}{overlay unbroken}{=\meta{graphical code}}{no default, initially unset}
  If the box is set to be \refKey{/tcb/breakable} but \emph{is not} broken actually
  or if the box is set to be \refKey{/tcb/unbreakable},
  then the \meta{graphical code} is added to the box drawing process.
  \refKey{/tcb/overlay} overwrites this key.
\end{docTcbKey}

\begin{docTcbKey}{overlay first}{=\meta{graphical code}}{no default, initially unset}
  If the box is set to be \refKey{/tcb/breakable} and \emph{is} broken actually,
  then the \meta{graphical code} is added to the box drawing process for
  the \emph{first} part of the break sequence.
  \refKey{/tcb/overlay} overwrites this key.
\end{docTcbKey}

\begin{docTcbKey}{overlay middle}{=\meta{graphical code}}{no default, initially unset}
  If the box is set to be \refKey{/tcb/breakable} and \emph{is} broken actually,
  then the \meta{graphical code} is added to the box drawing process for
  the \emph{middle} parts (if any) of the break sequence.
  \refKey{/tcb/overlay} overwrites this key.
\end{docTcbKey}

\begin{docTcbKey}{overlay last}{=\meta{graphical code}}{no default, initially unset}
  If the box is set to be \refKey{/tcb/breakable} and \emph{is} broken actually,
  then the \meta{graphical code} is added to the box drawing process for
  the \emph{last} part of the break sequence.
  \refKey{/tcb/overlay} overwrites this key.
\end{docTcbKey}

\begin{docTcbKey}{overlay unbroken and first}{=\meta{graphical code}}{no default, initially unset}
  This is an optimized abbreviation for setting
  \refKey{/tcb/overlay unbroken} and
  \refKey{/tcb/overlay first} together.
  \refKey{/tcb/overlay} overwrites this key.
\end{docTcbKey}

\begin{docTcbKey}{overlay middle and last}{=\meta{graphical code}}{no default, initially unset}
  This is an optimized abbreviation for setting
  \refKey{/tcb/overlay middle} and
  \refKey{/tcb/overlay last} together.
  \refKey{/tcb/overlay} overwrites this key.
\end{docTcbKey}



\begin{dispListing*}{breakable,before upper={This example demonstrates
the application of break sequence specific overlay options.
Here, we define an environment |myexample| based
on |tcolorbox| where the visible drawing is done totally by overlay keys.\par
Here, the first application of |myexample| produces an unbroken |tcolorbox|.
The frame is drawn by the code given with \refKey{/tcb/overlay unbroken}.\par
The second application of |myexample| is broken into several parts which
are drawn by the codes given with
\refKey{/tcb/overlay first}, \refKey{/tcb/overlay middle}, and
\refKey{/tcb/overlay last}.
\par\bigskip
}}
% Preamble:
%\usepackage{tikz,lipsum,calc}
%\tcbuselibrary{skins,breakable}
%\newcounter{example}
%\newlength{\examlen}
\colorlet{colexam}{red!75!black}

\newtcolorbox[use counter=example]{myexample}{%
  phantom={\settowidth{\global\examlen}{\Large\bfseries Example \thetcbcounter}},%
  title={Example \thetcbcounter},
  coltitle=colexam,fonttitle=\Large\bfseries,
  enhanced,breakable,before=\par\medskip,parbox=false,
  frame hidden,interior hidden,segmentation hidden,
  boxsep=0pt,left=0pt,right=3mm,toptitle=2mm,pad at break=0mm,
  overlay unbroken={\draw[colexam,line width=1pt] (frame.north west)
    --([xshift=-0.5pt]frame.north east)--([xshift=-0.5pt]frame.south east)
    --(frame.south west);
    \draw[colexam,line width=2pt] ([yshift=0.5pt]frame.north west)
    -- +(\examlen,0pt);},
  overlay first={\draw[colexam,line width=1pt] (frame.north west)
    --([xshift=-0.5pt]frame.north east)--([xshift=-0.5pt]frame.south east);
    \draw[red!75!black,line width=2pt] ([yshift=0.5pt]frame.north west)
    -- +(\examlen,0pt);},
  overlay middle={\draw[colexam,line width=1pt] ([xshift=-0.5pt]frame.north east)
    --([xshift=-0.5pt]frame.south east); },
  overlay last={\draw[colexam,line width=1pt] ([xshift=-0.5pt]frame.north east)
    --([xshift=-0.5pt]frame.south east)--(frame.south west);}%
}

\begin{myexample}
\lipsum[1]
\end{myexample}

\begin{myexample}
\lipsum[2-11]
\end{myexample}

\lipsum[12]% following text
\end{dispListing*}
{\tcbusetemp}


\begin{dispExample}
% \tcbuselibrary{skins}
% \newcounter{example}
\newtcolorbox[use counter=example]{FancyTitle}[3][]{%
  enhanced,colback=blue!10!white,colframe=orange,top=4mm,
  enlarge top by=\baselineskip/2+1mm,
  enlarge top at break by=0mm,pad at break=2mm,
  fontupper=\normalsize,label={#3},
  overlay unbroken and first={%
    \node[rectangle,rounded corners,draw=black,fill=blue!20!white,
      inner sep=1mm,anchor=west,font=\small]
      at ([xshift=4.5mm]frame.north west)
         {\strut\textbf{Example \thetcbcounter: #2}};},
  #1}%

\begin{FancyTitle}{My fancy title}{fancy:title}
  \lipsum[1]
\end{FancyTitle}
\end{dispExample}

\clearpage
\subsection{Floating Objects}
\begin{docTcbKey}{floatplacement}{=\meta{values}}{no default, initially \texttt{htb}}
  Sets \meta{values} as default values for the usage of |float|.
  Feasible are the usual parameters for floating objects.
\begin{dispListing}
\tcbset{enhanced,colback=red!5!white,colframe=red!75!black,
   watermark color=red!15!white}

\begin{tcolorbox}[floatplacement=t,float,
                  title=Floating box from |floatplacement|,
                  watermark text={I am floating}]
  This floating box is placed at the top of a page.
\end{tcolorbox}
\end{dispListing}
\end{docTcbKey}
{\tcbusetemp}


\begin{docTcbKey}{float}{\colOpt{=\meta{values}}}{default from \texttt{floatplacement}}
  Turns the box to a floating object where \meta{values} are the
  usual parameters for such floating objects.
  If they are not used, the placement uses the default values given by
 |floatplacement|.
\begin{dispListing}
\begin{tcolorbox}[float, title=Floating box from |float|,
    enhanced,watermark text={I'm also floating}]
  This box floats to a feasible place automatically. You do not have to
  use a numbering for this floating object.
\end{tcolorbox}
\end{dispListing}
\end{docTcbKey}
{\tcbusetemp}


\begin{docTcbKey}{nofloat}{}{style, initially set}
  Turns the floating behavior off.
\end{docTcbKey}


\clearpage
\subsection{Side by Side}\label{subsec:sidebyside}
\begin{marker}
  Further side by side options for code examples are
  \refKey{/tcb/listing side text},
  \refKey{/tcb/text side listing},
  \refKey{/tcb/listing outside text}, and
  \refKey{/tcb/text outside listing}.
\end{marker}

\begin{docTcbKey}{sidebyside}{\colOpt{=\meta{boolean value}}}{default |true|, initially |false|}
Normally, the upper part and the lower part of the box have their positions
as their names suggest. If |sidebyside| is set to |true|, the upper part
is drawn \emph{left-handed} and the lower part is drawn \emph{right-handed}.
Both parts are drawn together with the geometry settings of the upper part but the
space is divided horizontally according to the following options.
Colors, fonts, and box content additions are used individually.
The resulting box is unbreakable.

\begin{dispExample}
\tcbset{colback=red!5!white,colframe=red!75!black,fonttitle=\bfseries}

\begin{tcolorbox}[title=My title,sidebyside]
This is the upper (\textit{left-handed}) part.
\tcblower
This is the lower (\textit{right-handed}) part.
\end{tcolorbox}
\end{dispExample}
\end{docTcbKey}


\begin{docTcbKey}{sidebyside align}{=\meta{alignment value}}{no default, initially |center|}
  Sets the vertical alignment for the left-handed and right-handed part.
  Feasible values are |center|, |top|, and |bottom|.
\begin{dispExample}
\tcbset{colback=red!5!white,colframe=red!75!black,fonttitle=\bfseries,nobeforeafter,
  left=2mm,right=2mm,sidebyside,sidebyside gap=6mm,width=(\linewidth-2mm)/3}

\begin{tcolorbox}[adjusted title=center,sidebyside align=center]
This is a text which is too long for one line.
\tcblower
This is a short text.
\end{tcolorbox}\hfill
\begin{tcolorbox}[adjusted title=top,sidebyside align=top]
This is a text which is too long for one line.
\tcblower
This is a short text.
\end{tcolorbox}\hfill
\begin{tcolorbox}[adjusted title=bottom,sidebyside align=bottom]
This is a text which is too long for one line.
\tcblower
This is a short text.
\end{tcolorbox}
\end{dispExample}
\end{docTcbKey}

\clearpage
\begin{docTcbKey}{sidebyside gap}{=\meta{length}}{no default, initially |10mm|}
Sets the horizontal distance between the left-handed and right-handed part to
\meta{length}.
\begin{dispExample}
\tcbset{colback=red!5!white,colframe=red!75!black,fonttitle=\bfseries,nobeforeafter,
  sidebyside,width=(\linewidth-2mm)/2}

\begin{tcolorbox}[adjusted title=Wide gap,sidebyside gap=30mm]
This is a text which is too long for one line.
\tcblower
This is a short text.
\end{tcolorbox}\hfill
\begin{tcolorbox}[adjusted title=Narrow gap,sidebyside gap=1mm]
This is a text which is too long for one line.
\tcblower
This is a short text.
\end{tcolorbox}
\end{dispExample}
\end{docTcbKey}


\begin{docTcbKey}{lefthand width}{=\meta{length}}{no default, initially unset}
Sets the width of the left-handed part to the given \meta{length}.
\begin{dispExample}
\tcbset{colback=red!5!white,colframe=red!75!black,fonttitle=\bfseries}

\begin{tcolorbox}[title=My title,sidebyside,lefthand width=3cm]
This is the upper (\textit{left-handed}) part.
\tcblower
This is the lower (\textit{right-handed}) part.
\end{tcolorbox}
\end{dispExample}
\end{docTcbKey}

\enlargethispage*{1cm}
\begin{docTcbKey}{righthand width}{=\meta{length}}{no default, initially unset}
Sets the width of the right-handed part to the given \meta{length}.
\begin{dispExample}
\tcbset{colback=red!5!white,colframe=red!75!black,fonttitle=\bfseries}

\begin{tcolorbox}[title=My title,sidebyside,righthand width=3cm]
This is the upper (\textit{left-handed}) part.
\tcblower
This is the lower (\textit{right-handed}) part.
\end{tcolorbox}
\end{dispExample}
\end{docTcbKey}

\clearpage
\begin{docTcbKey}{lefthand ratio}{=\meta{fraction}}{no default, initially |0.5|}
Sets the width of the left-handed part to the given \meta{fraction} of
the available space. \meta{fraction} is a value between |0| and |1|.
\begin{dispExample}
\tcbset{colback=red!5!white,colframe=red!75!black,fonttitle=\bfseries}

\begin{tcolorbox}[title=My title,sidebyside,lefthand ratio=0.25]
This is the upper (\textit{left-handed}) part.
\tcblower
This is the lower (\textit{right-handed}) part.
\end{tcolorbox}
\end{dispExample}
\end{docTcbKey}


\begin{docTcbKey}{righthand ratio}{=\meta{fraction}}{no default, initially |0.5|}
Sets the width of the right-handed part to the given \meta{fraction} of
the available space. \meta{fraction} is a value between |0| and |1|.
\begin{dispExample}
\tcbset{colback=red!5!white,colframe=red!75!black,fonttitle=\bfseries}

\begin{tcolorbox}[title=My title,sidebyside,righthand ratio=0.25]
This is the upper (\textit{left-handed}) part.
\tcblower
This is the lower (\textit{right-handed}) part.
\end{tcolorbox}
\end{dispExample}
\end{docTcbKey}


\clearpage
\subsection{Embedding into the Surroundings}\label{subsec:surroundings}
Typically, but not necessarily, a |tcolorbox| is put inside a separate paragraph
and has some vertical space before and after it.
This behavior is controlled by the keys |before| and |after|.

\begin{docTcbKey}{before}{=\meta{code}}{no default, initially see \refKey{/tcb/autoparskip}}
  Sets the \meta{code} which is executed before the colored box.
  It is  not used for floating boxes.
\end{docTcbKey}

\begin{docTcbKey}{after}{=\meta{code}}{no default, initially see \refKey{/tcb/autoparskip}}
  Sets the \meta{code} which is executed after the colored box.
  It is not used for floating boxes.
\end{docTcbKey}

\begin{docTcbKey}{parskip}{}{style, no value}
  Sets the keys |before| and |after| to values which are
  recommended, if the package |parskip| \emph{is} used and there is no better
  idea for |before| and |after|.
\begin{dispListing}
\tcbset{parskip/.style={before={\par\pagebreak[0]\noindent},
                        after={\par}}}
\end{dispListing}
\end{docTcbKey}

\begin{docTcbKey}{noparskip}{}{style, no value}
  Sets the keys |before| and |after| to values which are
  recommended, if the package |parskip| is \emph{not} used and there is no better
  idea for |before| and |after|.
\begin{dispListing}
\tcbset{noparskip/.style={before={\par\smallskip\pagebreak[0]\noindent},
                          after={\par\smallskip}}}
\end{dispListing}
\end{docTcbKey}

\begin{docTcbKey}{autoparskip}{}{style, no value, initially set}
  Tries to detect the usage of the package |parskip| and sets
  the keys |before| and |after| accordingly. Actually, the following is done:
  \begin{itemize}
  \item If the length of |\parskip| is greater than |0pt| at the begin of the document,
    \refKey{/tcb/parskip} is executed. Here, the usage of package |parskip| is \emph{assumed}.
  \item Otherwise, if the length of |\parskip| is not greater than |0pt| at the begin of the document,
    \refKey{/tcb/noparskip} is executed. Here, the absence of package |parskip| is \emph{assumed}.
  \end{itemize}
  |autoparskip| is the default for the package |tcolorbox|, if |before| or |after|
  are not changed otherwise.
\end{docTcbKey}

\begin{docTcbKey}{nobeforeafter}{}{style, no value}
  Abbreviation for clearing the keys |before| and |after|. The colored box
  is not put into a paragraph and there is no space before or after the box.
\begin{dispExample}
\tcbset{myone/.style={colback=LightGreen,colframe=DarkGreen,
  equal height group=nobefaf,width=\linewidth/4,nobeforeafter}}
\begin{tcolorbox}[myone,title=Box 1]Box 1\end{tcolorbox}%
\begin{tcolorbox}[myone,title=Box 2]Box 2\end{tcolorbox}%
\begin{tcolorbox}[myone,title=Box 3]Box 3\end{tcolorbox}%
\begin{tcolorbox}[myone,title=Box 4]Box 4\end{tcolorbox}
\end{dispExample}
\end{docTcbKey}


\begin{docTcbKey}{baseline}{=\meta{length}}{no default, initially |0pt|}
  Used to set the |\pgfsetbaseline| value of the resulting |tcolorbox|.
\end{docTcbKey}


\clearpage
\subsection{Bounding Box}
Normally, every |tcolorbox| has a bounding box which fits exactly to the
dimensions of the outer frame. Therefore, \LaTeX\ reserves exactly the space
needed for the box.
This behavior can be changed by enlarging (or shrinking) the bounding box.
If the bounding box is enlarged, the |tcolorbox| will get some clearance
around it. If the bounding box is shrunk, i.\,e.\ enlarged with negative
values, the |tcolorbox| will overlap to other parts of the page.
For example, the |tcolorbox| could be stretched into the page margin.

\begin{docTcbKey}{enlarge top by}{=\meta{length}}{no default, initially |0mm|}
  Enlarges the bounding box distance to the top of the box by \meta{length}.
\begin{dispExample}
\tcbset{colframe=blue!75!black,colback=white}

\begin{tcolorbox}[enlarge top by=-5mm]
This is a \textbf{tcolorbox}.
\end{tcolorbox}
\begin{tcolorbox}[enlarge top by=5mm]
This is a \textbf{tcolorbox}.
\end{tcolorbox}
\end{dispExample}
\end{docTcbKey}



\begin{docTcbKey}{enlarge bottom by}{=\meta{length}}{no default, initially |0mm|}
  Enlarges the bounding box distance to the bottom of the box by \meta{length}.
\begin{dispExample}
\tcbset{colframe=blue!75!black,colback=white}

\begin{tcolorbox}[enlarge bottom by=5mm]
This is a \textbf{tcolorbox}.
\end{tcolorbox}
\begin{tcolorbox}[enlarge bottom by=-5mm]
This is a \textbf{tcolorbox}.
\end{tcolorbox}
\end{dispExample}
\end{docTcbKey}



\clearpage
\begin{docTcbKey}{enlarge left by}{=\meta{length}}{no default, initially |0mm|}
  Enlarges the bounding box distance to the left side of the box by \meta{length}.
\begin{dispExample}
\tcbset{colframe=blue!75!black,colback=white}

\begin{tcolorbox}[enlarge left by=2cm,width=\linewidth-2cm]
This is a \textbf{tcolorbox}.
\end{tcolorbox}
\begin{tcolorbox}[enlarge left by=-2cm,width=\linewidth+2cm]
This is a \textbf{tcolorbox}.
\end{tcolorbox}
\end{dispExample}
\end{docTcbKey}

\begin{docTcbKey}{enlarge right by}{=\meta{length}}{no default, initially |0mm|}
  Enlarges the bounding box distance to the right side of the box by \meta{length}.
\begin{dispExample}
\tcbset{colframe=blue!75!black,colback=white}

\begin{tcolorbox}[enlarge right by=-2cm,width=\linewidth+2cm]
This is a \textbf{tcolorbox}.
\end{tcolorbox}
\begin{tcolorbox}[enlarge right by=2cm,width=\linewidth-2cm]
This is a \textbf{tcolorbox}.
\end{tcolorbox}
\end{dispExample}
\end{docTcbKey}

\clearpage
\begin{docTcbKey}{toggle enlargement}{=\meta{toggle preset}}{default |evenpage|, initially |none|}
  According to the \meta{toggle preset}, the left and the right enlargements of
  the bounding box are switched or not. Feasible values are:
  \begin{itemize}
  \item |none|: no switching.
  \item |forced|: the values of the left and right enlargement are switched.
  \item |evenpage|: if the page is an even page, the values of the left and
    right enlargement are switched. It is recommended to use this setting in
    conjunction with \refKey{/tcb/check odd page}.
    %Note that the page number detection may
    %fail in certain cases. For these cases, use the |forced| value.
  \end{itemize}
\begin{dispExample}
% \usepackage{changepage} for 'check odd page'
\tcbset{colframe=blue!75!black,colback=white,
enlarge left by=-20mm,enlarge right by=5mm,width=\linewidth+15mm}

\begin{tcolorbox}[toggle enlargement=none]
This is a \textbf{tcolorbox}.
\end{tcolorbox}
\begin{tcolorbox}[toggle enlargement=forced]
This is a \textbf{tcolorbox}.
\end{tcolorbox}
\begin{tcolorbox}[toggle enlargement=evenpage,check odd page]
  This page is an \ifthenelse{\isodd{\thepage}}{odd}{even} page.
  Therefore, the left and right enlargements
  \ifthenelse{\isodd{\thepage}}{are not}{are} toggled (with some luck).
\end{tcolorbox}
\end{dispExample}

\begin{dispListing}
\begin{tcolorbox}[colframe=red!60!black,colback=red!15!white,
  fonttitle=\bfseries,title=Floating box from \texttt{toggle enlargement},
  width=\textwidth+20mm,enlarge right by=-20mm,
  toggle enlargement=evenpage,float=t]
  This page is an \ifthenelse{\isodd{\thepage}}{odd}{even} page.
  Therefore, the left and right enlargements
  \ifthenelse{\isodd{\thepage}}{are not}{are} toggled (with some luck; otherwise
  use |forced|). This box stretches to the right margin on odd pages and to the left
  margin on even pages. The current document is one-sided -- this feature makes
  sense for two-sided documents only.
\end{tcolorbox}
\end{dispListing}
\tcbusetemp
\end{docTcbKey}


\clearpage
\begin{marker}
The following keys should not be used with breakable boxes or boxes with a
lower part.
\end{marker}

\begin{docTcbKey}{shrink tight}{}{style, no value, initially unset}
  The total colored box is shrunk to the dimensions of the upper
  part. There should be no lower part and no title.
  This style sets the \refKey{/tcb/boxsep} to |0pt| and other geometry keys
  to fitting values. This option is likely to be used with the following
  extrusion keys.
\begin{dispExample}
\tcbset{colframe=blue!75!black,colback=white,arc=0mm,boxrule=0.4pt,
        nobeforeafter,tcbox raise base,shrink tight}

\begin{tcolorbox}
This is a \textbf{tcolorbox}.
\end{tcolorbox}

Lorem \tcbox{ipsum} dolor sit amet, consectetuer adipiscing elit.
\end{dispExample}
\end{docTcbKey}

\begin{docTcbKey}{extrude left by}{=\meta{length}}{style, no default, initially unset}
  The (upper part of the) colored box is extruded by the given \meta{length} to the left side.
  The inner width and the bounding box is kept unchanged and the operation
  is additiv!
\begin{dispExample}
\tcbset{enhanced,colframe=red,colback=yellow!25!white,
  frame style={opacity=0.25},interior style={opacity=0.5},
  nobeforeafter,tcbox raise base,shrink tight,extrude by=2mm}

Lorem ipsum dolor sit amet, consectetuer adipiscing elit. Ut purus elit,
vestibulum ut, placerat ac, adipiscing vitae, felis.
\tcbox[extrude left by=1cm]{Curabitur} dictum gravida mauris.
Nam arcu libero, nonummy eget, consectetuer id, vulputate a, magna.
\end{dispExample}
\end{docTcbKey}

\begin{docTcbKey}{extrude right by}{=\meta{length}}{style, no default, initially unset}
  The (upper part of the) colored box is extruded by the given \meta{length} to the right side.
  The inner width and the bounding box is kept unchanged and the operation
  is additiv!
\begin{dispExample}
\tcbset{enhanced,colframe=red,colback=yellow!25!white,
  frame style={opacity=0.25},interior style={opacity=0.5},
  nobeforeafter,tcbox raise base,shrink tight,extrude by=2mm}

Lorem ipsum dolor sit amet, consectetuer adipiscing elit. Ut purus elit,
vestibulum ut, placerat ac, adipiscing vitae, felis.
\tcbox[extrude right by=1cm]{Curabitur} dictum gravida mauris.
Nam arcu libero, nonummy eget, consectetuer id, vulputate a, magna.
\end{dispExample}
\end{docTcbKey}

\clearpage
\begin{docTcbKey}{extrude top by}{=\meta{length}}{style, no default, initially unset}
  The (upper part of the) colored box is extruded by the given \meta{length} to the top side.
  The inner width and the bounding box is kept unchanged and the operation
  is additiv!
\begin{dispExample}
\tcbset{enhanced,colframe=red,colback=yellow!25!white,
  frame style={opacity=0.25},interior style={opacity=0.5},
  nobeforeafter,tcbox raise base,shrink tight,extrude by=2mm}

Lorem ipsum dolor sit amet, consectetuer adipiscing elit. Ut purus elit,
vestibulum ut, placerat ac, adipiscing vitae, felis.
\tcbox[extrude top by=1cm]{Curabitur} dictum gravida mauris.
Nam arcu libero, nonummy eget, consectetuer id, vulputate a, magna.
\end{dispExample}
\end{docTcbKey}

\begin{docTcbKey}{extrude bottom by}{=\meta{length}}{style, no default, initially unset}
  The (upper part of the) colored box is extruded by the given \meta{length} to the bottom side.
  The inner width and the bounding box is kept unchanged and the operation
  is additiv!
\begin{dispExample}
\tcbset{enhanced,colframe=red,colback=yellow!25!white,
  frame style={opacity=0.25},interior style={opacity=0.5},
  nobeforeafter,tcbox raise base,shrink tight,extrude by=2mm}

Lorem ipsum dolor sit amet, consectetuer adipiscing elit. Ut purus elit,
vestibulum ut, placerat ac, adipiscing vitae, felis.
\tcbox[extrude bottom by=1cm]{Curabitur} dictum gravida mauris.
Nam arcu libero, nonummy eget, consectetuer id, vulputate a, magna.
\end{dispExample}
\end{docTcbKey}

\begin{docTcbKey}{extrude by}{=\meta{length}}{style, no default, initially unset}
  The (upper part of the) colored box is extruded by the given \meta{length} to all sides.
  The inner width and the bounding box is kept unchanged and the operation
  is additiv!
\begin{dispExample}
\tcbset{enhanced,colframe=red,colback=yellow!25!white,
  frame style={opacity=0.25},interior style={opacity=0.5},
  nobeforeafter,tcbox raise base,shrink tight,extrude by=2mm}

Lorem ipsum dolor sit amet, consectetuer adipiscing elit. Ut purus elit,
vestibulum ut, placerat ac, adipiscing vitae, felis. \tcbox{Curabitur} dictum
gravida mauris. \tcbox[colframe=Green,interior style={opacity=0.0}]{Nam}
arcu libero, nonummy eget, consectetuer id, \tcbox{vulputate} a, magna. Donec
vehicula augue eu neque. Pellentesque habitant morbi tristique senectus et netus
et malesuada fames ac turpis egestas. \tcbox{Mauris ut leo.}
\end{dispExample}
\end{docTcbKey}

\clearpage
\subsection{Layered Boxes and Every Box Settings}\label{subsec:everybox}
A |tcolorbox| may contain another |tcolorbox| and so on. The package
takes track of the nesting level using a counter |tcblayer|. Counter values
may be used for doing some fancy things, but you should never change
the counter value yourself.

The package takes special care for the first four layers or nesting levels,
called managed layers.
Here, footnote texts are administrated to find their intended place
and specific layer dependent options may be set by changing
\refKey{/tcb/every box on layer n}.
If needed, the number of managed layers can be increased by setting
\refCom{tcbsetmanagedlayer} to a higher value than 4.

The following styles have a considerable influence on how layered boxes
are processed. Note especially that nested boxes are getting a
\refKey{/tcb/reset} by default. You can change this, but be prepared for
suprises if you do.

If the defaults are \emph{not changed}, a |tcolorbox| gets its options
in the following order. Following options overwrite preceding options.
\begin{enumerate}
\item On package load, all options are set to default values.
\item Every \refCom{tcbset} command adds or changes options for the following boxes
  inside the current \TeX\ group.
\item While entering a |tcolorbox|, a \refKey{/tcb/every box on layer n} or
  \refKey{/tcb/every box on higher layers} option list is applied.
  With default settings this means:
  \begin{itemize}
  \item For layer 1 (lowest layer), the \refKey{/tcb/every box} option list is applied.
    Not overwritten options given by a preceding \refCom{tcbset} survive.
  \item For layer 2 and above (nested boxes), a \refKey{/tcb/reset} followed by
    \refKey{/tcb/every box} option list is applied.
    Every resettable options given by a preceding \refCom{tcbset}
    and by the sourrounding box(es) are reset.
  \end{itemize}
\item The \meta{options} given to the |tcolorbox| are applied.
  Or, if the box was generated by \refCom{newtcolorbox} or friends,
  the \meta{options} given there are applied.
\item If the box was generated by \refCom{newtcolorbox} or friends,
  some automated options are applied.
\end{enumerate}


\begin{docTcbKey}{every box}{}{style}
By default, this style is empty.
\begin{dispListing}
% default setting:
\tcbset{every box/.style={}}
\end{dispListing}
It may be changed by redefining this style.
\begin{dispListing}
% setting all boxes to be enhanced:
\tcbset{every box/.style={enhanced}}
\end{dispListing}
\end{docTcbKey}

\clearpage
\begin{docTcbKey}{every box on layer n}{}{style}
Here, |n| has to be replaced by a number ranging from 1 to the highest
managed layer number (4 by default).
\begin{dispListing}
% default settings:
\tcbset{
  every box on layer 1/.style={every box},
  every box on layer 2/.style={reset,every box},
  every box on layer 3/.style={reset,every box},
  every box on layer 4/.style={reset,every box},
  }
\end{dispListing}
\end{docTcbKey}


\begin{docTcbKey}{every box on higher layers}{}{style}
Higher layers are layers above the highest
managed layer number (4 by default).
\begin{dispListing}
\tcbset{every box on higher layers/.style={reset,every box}}
\end{dispListing}
\end{docTcbKey}


\begin{docCommand}{tcbsetmanagedlayer}{\marg{number}}
Replaces the highest managed layer number by \meta{number} where 4 is
the default. This macro can only be used inside the preamble.
Using a \meta{number} lower than 4 typically makes no sense, but is
not forbidden.
\end{docCommand}

\begin{tcboutputlisting}
% \usepackage{lipsum}
% \tcbuselibrary{skins,breakable}
\tcbset{enhanced,breakable,colframe=red!75!black,fonttitle=\bfseries,
  colback=red!5!white,
  every box/.style={enhanced,watermark text=\thetcblayer,
    before=\par\smallskip,after=\par\smallskip},
  every box on layer 2/.style={reset,every box,colback=yellow!10!white,
    drop fuzzy shadow}}
\begin{tcolorbox}[title=Layer 1 Box]
Here comes a footnote\footnote{Footnote from layer 1 box}.
\lipsum[2]
  \begin{tcolorbox}[title=Layer 2 Box]
  abc\footnote{The footnote of abc}
  \end{tcolorbox}
  \begin{tcolorbox}[title=Another Box,ams equation]
    \tcbhighmath{\sum\limits_{n=1}^{\infty} \frac{1}{n}} = \infty.
  \end{tcolorbox}
Some text\footnote{Footnote from some text}.
  \begin{tcolorbox}[title=Yet Another Box]
    \tcboxfit[height=2cm]{\lipsum[1]}
    My text.
    \begin{tcolorbox}
      Another lipsum text\footnote{A lipsum text}. \lipsum[3]
      \begin{tcolorbox}[title=Layer 4,colframe=blue,colback=white]
        Layer 4\footnote{Layer 4 footnote}
      \end{tcolorbox}
      The End\footnote{Last footnote}.
    \end{tcolorbox}
  \end{tcolorbox}
\end{tcolorbox}
\end{tcboutputlisting}

\tcbinputlisting{base example,listing only,listing style=mydocumentation}

{\tcbuselistingtext}


\clearpage
\subsection{Capture Mode}\label{subsec:capture}

\begin{docTcbKey}{capture}{=\meta{mode}}{no default, initially |minipage|}
  The capture \meta{mode} defines how the box content is processed.

Feasible values for \meta{mode} are:
\begin{itemize}
\item |minipage|:\\
  This is the default \meta{mode} for \refEnv{tcolorbox}.
  The content may have an upper and a lower part. Optionally, the box
  can be \refKey{/tcb/breakable}. The box content is put into a
  minipage or into something similar to a minipage.
\item |hbox|:\\
  This is the default \meta{mode} for \refCom{tcbox}. The content cannot have
  a lower part and cannot be broken. The colored box is sized according
  to the dimensions of the content.
  A shortcut to set this mode is \refKey{/tcb/hbox}.
\item |fitbox|: (needs the \mylib{fitting} library)\\
  This is the default \meta{mode} for \refCom{tcboxfit}. The content cannot have
  a lower part and cannot be broken.
  The content is sized according to the dimensions of the colored box.
  A shortcut to set this mode is \refKey{/tcb/fit}.
\end{itemize}

\begin{dispExample}
\tcbset{colframe=blue!75!black,colback=white}

\begin{tcolorbox}[capture=minipage]
This is a tcolorbox.
\end{tcolorbox}

\begin{tcolorbox}[capture=hbox]
This is a tcolorbox.
\end{tcolorbox}

\begin{tcolorbox}[capture=fitbox,height=9mm]% needs the 'fitting' library
This is a tcolorbox.
\end{tcolorbox}
\end{dispExample}
\end{docTcbKey}



\begin{docTcbKey}{hbox}{}{style, no default}
  Shortcut for |capture=hbox|.
\begin{dispExample}
\tcbset{colframe=blue!75!black,colback=white}

\begin{tcolorbox}[hbox]
This is a tcolorbox.
\end{tcolorbox}
\end{dispExample}
\end{docTcbKey}


\begin{docTcbKey}{minipage}{}{style, no default}
  Shortcut for |capture=minipage|.
\end{docTcbKey}


\clearpage
\subsection{Text Characteristics}
\begin{docTcbKey}{parbox}{\colOpt{=\meta{boolean value}}}{default |true|, initially |true|}
  The text inside a |tcolorbox| is formatted using a \LaTeX\ |minipage|
  if the box is unbreakable. If breakable, the box tries a mimicry of
  a |minipage|. In a |minipage| or |parbox|, paragraphs are formatted slightly different
  as the main text. If the key value is set to |false|, the normal main text
  behavior is restored. In some situations, this has some unwanted side
  effects. It is recommended to use this experimental setting only
  where you really want to have this feature.
\end{docTcbKey}

\begin{dispListing}
% \usepackage{lipsum}  % preamble
\tcbset{width=(\linewidth-2mm)/2,nobeforeafter,arc=1mm,
  colframe=blue!75!black,colback=white,fonttitle=\bfseries,fontupper=\small,
  left=2mm,right=2mm,top=1mm,bottom=1mm,equal height group=parbox}

\begin{tcolorbox}[parbox,adjusted title={parbox=true (normal)}]
  \lipsum[1-2]
\end{tcolorbox}\hfill%
\begin{tcolorbox}[parbox=false,adjusted title={parbox=false}]
  \lipsum[1-2]
\end{tcolorbox}%
\end{dispListing}
{\tcbusetemp}

\subsection{Files}
\begin{docTcbKey}{tempfile}{=\meta{file name}}{no default, initially \cs{jobname.tcbtemp}}
  Sets \meta{file name} as name for the temporary file which is used inside
  \refEnv{tcbwritetemp} and \refCom{tcbusetemp} implicitely.
\end{docTcbKey}

\subsection{\texttt{\textbackslash tcbox} Specials}
The following options are applicable for \refCom{tcbox} and \refCom{tcboxmath}
only.

\begin{docTcbKey}{tcbox raise}{=\meta{length}}{no default, initially \texttt{0pt}}
  Raises the \refCom{tcbox} by the given \meta{length}.
 % Sets the line width of the right rule to \meta{length}.
\begin{dispExample}
\tcbset{colframe=blue!50!black,colback=white,colupper=red!50!black,
        fonttitle=\bfseries,before=,after=,center title}

Test\dotfill
\tcbox[tcbox raise base]{Hello World 1}\dotfill
\tcbox{Hello World 2}\dotfill
\tcbox[tcbox raise=5mm]{Hello World 3}
\end{dispExample}
\end{docTcbKey}

\begin{docTcbKey}{tcbox raise base}{}{style, no value, initially unset}
  Raises the \refCom{tcbox} such that the base of its content matches
  the base of the environmental line; see example above.
\end{docTcbKey}


\subsection{Skins}
There are additional option keys which change the appearance of a |tcolorbox|.
If only the core package is used, there is only one \emph{skin} and these
keys are meaningless.
The library \mylib{skins} adds more skins. The appropriate option keys for skins of
the core package are therefore described in section \ref{sec:skincorekeys} from
page \pageref{sec:skincorekeys}.

\clearpage
\subsection{Counters, Labels and References}

\begin{docTcbKey}{phantom}{=\meta{code}}{no default, initially unset}
The \meta{code} is put in a box at the upper left corner of the |tcolorbox|.
If the |tcolorbox| is breakable, the \meta{code} is executed for the first box of
the break sequence only. If there already was some phantom code given, the
new \meta{code} is appended.\par
The \meta{code} is intended to be used for counter stepping, labelling, and
related operations which do not produce visible text.
\begin{itemize}
\item The \meta{code} is executed before the title and box content, i.\,e.\ counter
  values are ensured to be increased before usage.
\item Labels are ensured to reference the correct page number.
\item The \meta{code} is executed only once even during fitting operations for
  title and box content.
\item In combination with the |hyperref| package, the hyper anchor is set
  to the upper left corner of the |tcolorbox|, i.\,e.\ links inside the pdf document
  will jump to the box pleasantly.
\item Since the \meta{code} is executed inside a \TeX\ group, only global
  operations can survive this group.
\end{itemize}
Examples for the |phantom| usage are given in Section \ref{listing:exercises}
from page \pageref{listing:exercises}, e.\,g.\
Example \ref{exe:tabular_example} on page \pageref{exe:tabular_example}.
\end{docTcbKey}

\begin{docTcbKey}{nophantom}{}{no value, initially set}
Removes the phantom code if set before.
\end{docTcbKey}

\begin{docTcbKey}{label}{=\meta{marker}}{no default, initially unset}
The \meta{marker} is set as label text for a reference with the |\ref| macro.
Typically, this option is used for numbered boxes, see Subsection \ref{sec:numberedboxes}
from page \pageref{sec:numberedboxes}, e.\,g.\ \refKey{/tcb/new/auto counter}.
\end{docTcbKey}

\begin{docTcbKey}{label type}{=\meta{type}}{no default, initially unset}
This option key can be used only in conjunction with the |cleveref| package
\cite{cubitt:2013a} which has to be loaded separately.
\meta{type} has to be a cross-reference type \emph{known} to |cleveref|
like |theorem|, |algorithm|, |result|, etc. References made with |cleveref|
will use this type. Note that using |label type| will result in compilation
errors, if |cleveref| is not loaded.
For an example, see \Vref{theo:meanvaluetheorem}.
\end{docTcbKey}

\begin{docTcbKey}{no label type}{}{no value, initially set}
Removes a \refKey{/tcb/label type}, if set before.
\end{docTcbKey}

\begin{docTcbKey}{step}{=\meta{counter}}{no default, initially unset}
Shortcut for |phantom={\refstepcounter{#1}}|. The given \meta{counter} is
increased and ready for labelling. This option is not needed when
using the convenient automated numbering introduced with version 2.40,
see Subsection \ref{sec:numberedboxes}
from page \pageref{sec:numberedboxes}.
\end{docTcbKey}

\begin{docTcbKey}{step and label}{=\marg{counter}\marg{marker}}{no default, initially unset}
Shortcut for using \refKey{/tcb/step} and \refKey{/tcb/label}. This option is not needed when
using the convenient automated numbering introduced with version 2.40,
see Subsection \ref{sec:numberedboxes}
from page \pageref{sec:numberedboxes}.
\end{docTcbKey}


\clearpage
\begin{docTcbKey}{list entry}{=\meta{text}}{no default, initially unset}
If the \flqq list of tcolorbox(es)\frqq\ feature described in Subsection
\ref{sec:listsof} from page \pageref{sec:listsof} is used, this key
describes the \meta{text} for an entry into the generated list, e.\,g.
\begin{dispListing}
list entry={\numberline{\thetcbcounter}My beautiful Example}
\end{dispListing}
See Section \ref{listing:exercises} from page \pageref{listing:exercises}
for a complete example.
\end{docTcbKey}

\begin{docTcbKey}{add to list}{=\marg{list}\marg{type}}{no default, initially unset}
If the \flqq list of tcolorbox(es)\frqq\ feature described in Subsection
\ref{sec:listsof} from page \pageref{sec:listsof} is used, list entries are
generated automatically. With this key, you can enforce an entry to the
given \meta{list} with the given \meta{type}.
This issues:\\
|\addcontentsline|\marg{list}\marg{type}\marg{entry text}
\end{docTcbKey}


\begin{docTcbKey}{check odd page}{\colOpt{=\meta{boolean value}}}{default |true|, initially |false|}
If set to |true|, the even/odd page testing from the package |changepage|
is applied. The \refKey{/tcb/toggle enlargement} check
and the  \refKey{/tcb/toggle left and right} check
will use the
|\ifoddpage| macro from this package.
This options is independent from \refKey{/tcb/phantom}.
Note that you have to include the package |changepage| by hand\footnote{If |changepage| is
not included, a compilation error will arise.}:

\begin{dispListing}
\usepackage{changepage}
\strictpagecheck
\end{dispListing}

\medskip
The macro |\ifoddpage| can be used inside overlay or watermark code to test if the
box is on an odd page. This will work also for boxes in a break sequence.
Note that you cannot use the test inside the normal
box content.

\begin{dispExample}
% \usepackage{changepage}
\tcbset{colframe=blue!75!black,colback=white}

\begin{tcolorbox}[enhanced,check odd page,
  watermark text={\ifoddpage Odd\else Even\fi\ page!}]
\lipsum[1]
\end{tcolorbox}
\end{dispExample}
\end{docTcbKey}


\clearpage
\subsection{Externalization}
If the \emph{externalization} library of the \texttt{tikz} package is used
and \refKey{/tcb/graphical environment} is set to |tikzpicture|,
a |tcolorbox| could trigger the externalization process which will arise
a compilation error.

To avoid this, there are two possible strategies:
\begin{itemize}
\item Ensure, that |\tikzexternaldisable| is set before a |tcolorbox| is used.
  If you typically use the pattern |\tikzexternalenable| \textit{some picture} |\tikzexternaldisable|,
  there is nothing to care about.
\item If \emph{externalization} is enabled globally, use \refKey{/tcb/shield externalize} to
  shield any |tcolorbox|. The preamble code could look like this:
\begin{dispListing}
\usetikzlibrary{external}
\tikzexternalize
\tcbset{shield externalize}
\end{dispListing}
\end{itemize}

\begin{docTcbKey}{shield externalize}{\colOpt{=\meta{boolean value}}}{default |true|, initially |false|}
If set to |true|, the drawing part of the |tcolorbox| is not being externalized
which is a good thing at the current state of art. Nevertheless, if the
|tcolorbox| contains a |tikzpicture|, this picture is still externalized.
Pictures drawn with help of \refKey{/tcb/tikz upper} or alike are \emph{not}
externalized.
\end{docTcbKey}

\begin{marker}
If a |tcolorbox| is used inside a node of an encircling |tikzpicture| which is externalized,
do \emph{not} use |\tikzexternaldisable| in front of the |tcolorbox|.
\refKey{/tcb/shield externalize} is deactivated automatically inside a |tikzpicture|.
\end{marker}



\subsection{Miscellaneous}
\begin{docTcbKey}{reset}{}{no value, initially set}
Sets (nearly) all |tcolorbox| settings (including loaded libraries) back to their default values.
\refKey{/tcb/savedelimiter} and \refKey{/tcb/capture} keep their values.
This option is useful for boxes in boxes where the inner box should not inherit
the settings of the outer box.
Note that for boxes inside boxes the |reset| is done automatically, if the
standard settings of the package are used (v2.40 and above), see
Section \ref{subsec:everybox} from page \pageref{subsec:everybox}.
%See \refCom{tcbhighmath} for an example.
\end{docTcbKey}
